With the aim of showcasing the tools used throughout the data acquirement process, as well as providing a user-friendly testing experience on the better models, a web application has been developed using popular toolkits and frameworks, namely SQLAlchemy and Django.

Web applications are typically comprised of three core components:
\begin{compactitem}[\textbullet]
    \item The \textbf{backend}, often powered by frameworks like Django, handles the business logic, user requests processing, database communication. It manages the application's functionality, authentication, security, and \acs{api} endpoints. 
    \item The \textbf{frontend}, on the other hand, focuses on the user interface, utilizing technologies such as \acs{html}, \acs{css}, \acl{js} and, in the case of Django, the Jinja2 template engine, to create visually appealing and interactive experiences.
    \item The \textbf{database}, powered by management systems like PostgreSQL or MySQL, stores and manipulates the application's data, providing efficient data storage, retrieval, and management.
\end{compactitem}
