Recommender systems can be classified into different types based on their techniques and focus. These types include:
\begin{compactitem}[\textbullet]
    \item \textbf{General Recommenders}, consisting of various techniques including collaborative filtering algorithms, which analyze user-item interactions to identify similar users or items, or \textbf{content-based} systems, which focus more on items' characteristics and attributes (e.g.\ category, language, author\ldots), analyzing their content to suggest similar items according to the user's preferences. Non-personalized methods, such as popularity-based recommendation, also fall into this category~\cite{CHAPTER:RS-HANDBOOK-GENERAL}.
    \item \textbf{Context-Aware Recommenders}, which incorporate contextual information, such as time, location, demographics, and user behavior, to provide personalized recommendations that adapt to the user's current situation. Additionally, these recommenders could also be \textbf{content-based}, incorporating relevant content as embeddings, for instance, to increase the amount of useful training data~\cite{CHAPTER:RS-HANDBOOK-CONTEXT}.
    \item \textbf{Sequential Recommenders}, which consider the temporal order of user interactions and exploit sequential patterns to make recommendations. They utilize historical sequences of user actions to predict the next item of interest~\cite[Chapter 3.5]{CHAPTER:RS-HANDBOOK-MUSIC}.
    \item \textbf{Knowledge-Based Recommenders}, which make use of external knowledge sources, such as knowledge graphs or semantic networks, to enhance the recommendation process. They utilize domain-specific knowledge and semantic relationships to generate more accurate and meaningful recommendations~\cite{CHAPTER:RS-HANDBOOK-KNOWLEDGE}.
\end{compactitem}
Hybrid approaches that combine multiple types are also common, allowing for leveraging different techniques to improve recommendations. The ultimate choice of model depends on the recommendation problem, available data, desired personalization and contextual relevance.