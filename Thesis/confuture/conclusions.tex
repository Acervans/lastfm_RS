This project has helped shed light on the competence of advanced recommendation algorithms in discerning latent factors and similarities from different track features. Remarkably, the manual assignment of tags by users proved to be the most influential in identifying representative similarities. Therefore, it should come as no surprise that \acs{vad} scores are likely contributors to recommendation models, as both tags and sentiment attributes were derived from manual annotations by human beings, with the latter stemming from precise and reliable methodologies. In spite of all the information lost by averaging and using objective text for a task that required subjectivity, relevance was still found in the extracted sentiment. This discovery emphasizes the dual significance of tags, providing a logical search space, and their sentiment attributes, offering an emotional dimension which, as evidenced in this work, was indeed significant in shaping musical preferences.

Delving into sentiment analysis and \acs{nlp}, despite the initial challenges, turned out to be highly captivating, as addressing the intricacies of sentiment attributes in music demanded innovative problem-solving; from the point of view of an undergraduate student, that is. The vast potential of sentiment analysis evoked an interest to further explore and contribute to its ongoing development and application in diverse fields.

In summary, this project served as a testament to the acquired knowledge and skills in web development, software engineering, and machine learning over the years; as well as a platform for exploring uncharted territories, namely \acs{nlp}, sentiment analysis, web scraping, and music recommendation, thus broadening the horizons of expertise. Ultimately, this Bachelor Thesis has not only contributed to the understanding of new and novel topics, but has also enriched the journey of intellectual growth and personal development.