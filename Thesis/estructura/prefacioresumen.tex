Existen varios elementos previos al texto como son el abstract, el resumen, los agradecimientos y el prefacio. Suele ser habitual que todos ellos sean obligatorios salvo el prefacio. Sin embargo no aparecerán los que no se indiquen usando los comandos correspondientes. Todos ellos deberán escribirse en ficheros aparte y el parámetro de los comandos será el fichero (incluyendo el \textit{path} relativo) sin el `.tex' del final. Los comandos son los siguientes:

\begin{description}
  \item [\textbackslash prefacefile\{fichero\}] Prefacio.
  \item [\textbackslash resumenfile\{fichero\}] Resumen en castellano.
  \item [\textbackslash abstractfile\{fichero\}] Resumen en inglés.
  \item [\textbackslash ackfile\{fichero\}] Agradecimientos.
\end{description}

Todos estos comandos deben usarse en el preámbulo, antes de \textbf{\textbackslash begin\{document\}}.

Así mismo se deben usar otros dos comandos para introducir las palabras clave al final del resumen y del abstract. Ambos comandos tienen un sólo parámetro y son las palabras clave separadas por comas. Estos comandos son \textbf{\textbackslash palabrasclave\{\}} y \textbf{\textbackslash keywords\{\}}. Estos comandos deben usarse al final del fichero con el resumen o con el abstract según corresponda.
