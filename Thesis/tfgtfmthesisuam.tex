% arara: clean: {files: [tfgtfmthesisuam.aux, tfgtfmthesisuam.idx, tfgtfmthesisuam.ilg, tfgtfmthesisuam.ind, tfgtfmthesisuam.bbl, tfgtfmthesisuam.bcf, tfgtfmthesisuam.blg, tfgtfmthesisuam.run.xml, tfgtfmthesisuam.fdb_latexmk, tfgtfmthesisuam.fls, tfgtfmthesisuam.loe, tfgtfmthesisuam.lof, tfgtfmthesisuam.lol, tfgtfmthesisuam.lot, tfgtfmthesisuam.ltb, tfgtfmthesisuam.out, tfgtfmthesisuam.toc, tfgtfmthesisuam.upa, tfgtfmthesisuam.upb, tfgtfmthesisuam.acn, tfgtfmthesisuam.acr, tfgtfmthesisuam.alg, tfgtfmthesisuam.glg, tfgtfmthesisuam.glo, tfgtfmthesisuam.gls, tfgtfmthesisuam.glsdefs, tfgtfmthesisuam.idx,  tfgtfmthesisuam.ilg, tfgtfmthesisuam.xdy, tfgtfmthesisuam.loa, tfgtfmthesisuam.gnuploterrors , tfgtfmthesisuam.mw, tfgtfmthesisuam.fdb_latexmk ]}
% arara: pdflatex: {shell: yes}
% arara: makeglossaries
% arara: makeindex: {style: tfgtfmthesisuam.ist }
% arara: bibtex
% arara: pdflatex: {shell: yes}
% arara: pdflatex: {shell: yes}
% arara: clean: {files: [tfgtfmthesisuam.aux, tfgtfmthesisuam.idx, tfgtfmthesisuam.ilg, tfgtfmthesisuam.ind, tfgtfmthesisuam.bbl, tfgtfmthesisuam.bcf, tfgtfmthesisuam.blg, tfgtfmthesisuam.run.xml, tfgtfmthesisuam.fdb_latexmk, tfgtfmthesisuam.fls, tfgtfmthesisuam.loe, tfgtfmthesisuam.lof, tfgtfmthesisuam.lol, tfgtfmthesisuam.lot, tfgtfmthesisuam.ltb, tfgtfmthesisuam.out, tfgtfmthesisuam.toc, tfgtfmthesisuam.upa, tfgtfmthesisuam.upb, tfgtfmthesisuam.acn, tfgtfmthesisuam.acr, tfgtfmthesisuam.alg, tfgtfmthesisuam.glg, tfgtfmthesisuam.glo, tfgtfmthesisuam.gls, tfgtfmthesisuam.glsdefs, tfgtfmthesisuam.idx,  tfgtfmthesisuam.ilg, tfgtfmthesisuam.xdy, tfgtfmthesisuam.loa, tfgtfmthesisuam.gnuploterrors , tfgtfmthesisuam.mw, tfgtfmthesisuam.fdb_latexmk ]}


\documentclass[english,epsbased,copyright,final,printable,covers,extendedindex,firstnumbered,tfg,gnuplot]{tfgtfmthesisuam}

\advisor{Alejandro Bellogín}
\levelin{Ingeniería Informática}
\title{Application of sentiment analysis on music recommendation}
\subtitle{Sentiment features efficiency in a hybrid music recommender}
\author{Javier Wang Zhou}
\privateaddress{C\textbackslash\ Francisco Tomás y Valiente Nº 11}
\copyrightdate{3 de Noviembre de 2017}

\dedication{A mi mujer y a mis hijos}
\famouscite{Lo peor es cuando has terminado un capítulo\\y la máquina de escribir no aplaude. \\[0.1em] \begin{flushright}Orson Welles\end{flushright}}
\prefacefile{inicio/prefacio}
\ackfile{inicio/agradecimientos}
\resumenfile{inicio/resumen}
\abstractfile{inicio/abstract}

\keywords{Algunas}
\palabrasclave{Otras}

\coverdata
{
  Escuela Politécnica Superior \\
  Universidad Autónoma de Madrid \\
  C\textbackslash Francisco Tomás y Valiente nº 11
}

\bibliographyconfig{tfgtfmthesisuam}

\datadir{data}
\graphicsdir{img}
\logosdir{img}
\codesdir{codes}


% hola
% lo de la estructura, suele ser relativamente estándar, lo mejor es mirar otros tfgs:
% https://abellogin.github.io/teaching.html
% por ejemplo el de Eric de hace unos años podría ser un buen punto de partida, aunque depende del tema del TFG
% puedes empezar con esto y cuando tengas una propuesta me escribes y lo reviso
% a partir de ahí, es centrarte en escribir las secciones que veas más claras, me avisas cuando están, yo te doy feedback y así vamos iterando
% lo más rollo siempre es el capítulo 2, el del estado del arte
% ya luego, el capítulo de desarrollo y diseño, como es expllicar lo que has hecho, cuesta menos y vas más rápido
% lo que hay que tener en cuenta en la parte de diseño es que se intenta demostrar que eres ingeniero, es decir, que sabes usar diagramas y eso que te enseñan en INGS
% ya luego va el capítulo de pruebas, donde habrá que explicar todo lo que me has enseñado hoy (o casi todo)
% ahí veremos cómo formatear los resultados, en qué orden presentarlo, y ver si tenemos explicación para lo que pasa
% y tampoco mucho más: como tú has dicho, abstract, intro y conclusiones se suelen dejar para el final, pero por supuesto si algún día estás inspirado, puedes escribir por ahí alguna idea

% pero si de aquí a lunes-martes-miércoles tienes la estructura, ya irías bien, porque eso te permite ir avanzando en paralelo (experimentos y escritura)

% sí, eso es: a nivel 1 o 2, es decir: nombre de sección y, cuando lo tengas claro, de subsección
% ya que los capítulos más o menos vienen definidos por la normativa

% si, en mayo es mejor asegurar que apruebas las asignaturas
% pero todo lo que adelantes, pues mejor, claro, pero la prioridad debe ser que apruebes, ya que si suspendes y entregas el tfg, no te dejan defender


\begin{document}

\chapter{State of the art\label{CAP:STATE_OF_THE_ART}}{sota/sota}
  \section{Recommender systems\label{SEC:RECSYS}}{sota/recsys/recsys}
    \subsection{Model classification\label{SS:MODELCLASS}}{sota/recsys/modelclass}
    \subsection{Context-based recommendation\label{SS:CONTEXTREC}}{sota/recsys/contextrec}
    \subsection{Music recommendation\label{SS:MUSICREC}}{sota/recsys/musicrec}
    \subsection{Evaluation\label{SS:EVALUATION}}{sota/recsys/evaluation}
  \section{Natural Language Processing\label{SEC:NLP}}{sota/nlp/nlp}
    \subsection{Sentiment Analysis\label{SS:SENTANALYSIS}}{sota/nlp/sentanalysis}
    \subsection{Word Embeddings\label{SS:WORDEMB}}{sota/nlp/wordemb}
  \section{Web applications\label{SEC:WEBAPP}}{sota/webapp/webapp}
    \subsection{MVT Architecture\label{SS:MVT}}{sota/webapp/mvt}
    \subsection{Django Framework\label{SS:DJANGO}}{sota/webapp/django}
  \section{Database management\label{SEC:DBM}}{sota/dbm/dbm}
    \subsection{RDBMS: PostgreSQL\label{SS:POSTGRESQL}}{sota/dbm/postgresql}
    \subsection{SQLAlchemy Toolkit\label{SS:SQLALCHEMY}}{sota/dbm/sqlalchemy}

\chapter{Design and implementation\label{CAP:DESIMP}}{desimp/desimp}
  \section{Web application\label{SEC:WEBAPP_DESIMP}}{desimp/webapp/webapp}
    \subsection{Requirements\label{SS:WEBAPP_REQ}}{desimp/webapp/requirements}
    \subsection{Design\label{SS:WEBAPP_DES}}{desimp/webapp/design}
    \subsection{Implementation\label{SS:WEBAPP_IMP}}{desimp/webapp/implementation}
  \section{Database\label{SEC:DATABASE_DESIMP}}{desimp/database/database}
    \subsection{Requirements\label{SS:DATABASE_REQ}}{desimp/database/requirements}
    \subsection{Design\label{SS:DATABASE_DES}}{desimp/database/design}
    \subsection{Implementation\label{SS:DATABASE_IMP}}{desimp/database/implementation}
  \section{Dataset\label{SEC:DATASET_DESIMP}}{desimp/dataset/dataset}
    \subsection{Requirements\label{SS:DATASET_REQ}}{desimp/dataset/requirements}
    \subsection{Design\label{SS:DATASET_DES}}{desimp/dataset/design}
    \subsection{Implementation\label{SS:DATASET_IMP}}{desimp/dataset/implementation}

\chapter{Tests and results\label{CAP:TESTRES}}{testres/testres}
  \section{Testing environment\label{SEC:TESTENV}}{testres/testenv}
  \section{Experiments\label{SEC:EXPERIMENTS}}{testres/experiments/experiments}
    \subsection{Data used in experiments\label{SS:DATAEXP}}{testres/experiments/dataused}
    \subsection{Testing libraries and model selection\label{SS:LIBMODELS}}{testres/experiments/libmodels}
    \subsection{Feature selection\label{SS:FEATSELECT}}{testres/experiments/featselect}
    \subsection{Integration of sentiment attributes\label{SS:SENTINTEG}}{testres/experiments/sentinteg}
    \subsection{Summary and performance analysis\label{SS:SUMANALYSIS}}{testres/experiments/sumanalysis}

\chapter{Conclusions and future work\label{CAP:CONFUTURE}}
  \section{Conclusions\label{SEC:CONCLUSIONS}}{confuture/conclusions}
  \section{Future Work\label{SEC:FUTUREWORK}}{confuture/futurework}

\end{document}
