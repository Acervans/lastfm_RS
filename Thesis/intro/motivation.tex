Music is inherently emotional and has the ability to evoke strong feelings in listeners; therefore, understanding the sentiment of music could provide valuable insights into the emotional characteristics and appeal of different songs. By applying sentiment analysis on music content, we could identify the emotional factors of tracks, such as the level of happiness, sadness, excitement, or relaxation. This information could be used to tailor recommendations based on the user's current emotional state or mood, providing a more personalized and engaging music experience.

Additionally, sentiment analysis could potentially assist in addressing the cold-start problem in recommendation, where limited user data is available. By analyzing the sentiment of music tracks, we could bridge the gap between user choice and music attributes, allowing for effective recommendations even for new users with sparse data.

By embedding sentiment into music recommendation, we aim at measuring the effectiveness of emotion, and the role it plays, in the representation of user preferences and the task of generating suitable candidates that cater for such preferences. Thus, we contribute to the development of music recommenders that not only consider user-item similarities or contextual information, but also incorporate the emotional aspects of music.