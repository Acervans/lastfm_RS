\subsubsection{Functional requirements}
\newcounter{DSF}
\begin{functionalmod}[DS]
    \item The dataset will be static, meaning that no new data will be added at any point, as it should be stable to provide consistent results for evaluation purposes.
    \item The dataset must include data from a minimum of 50,000 users to ensure a substantial user base, necessary for a reasonable evaluation of the recommendation models.
    \begin{functionalmod}[DS]
        \item Each user will have associated recent, top and loved tracks, as well as top artists and top albums, to capture diverse preferences and generate more extensive user profiles.
    \end{functionalmod}
    \item The dataset will include tags associated with each track to perform sentiment analysis on.
    \item The dataset will include additional information about tracks, namely artists and albums, to enable a fair comparison between the embedding of sentiment attributes and other track features.
    \begin{functionalmod}[DS]
        \item Artists and albums will include their own associated tags to expand on tag data, contributing to track tagging and reduction of data sparsity.
    \end{functionalmod}
    \setcounter{DSF}{\value{enumi}}
\end{functionalmod}
\subparagraph{Sentiment analyzer}
\begin{functionalmod}[DS]
    \setcounter{enumi}{\value{DSF}}
    \item The sentiment analyzer will generate sentiment attributes, specifically valence, arousal, dominance, and sentiment ratio or proportion of positive to negative words, for any given English text.
    \item The sentiment analyzer must load all necessary information into memory during initialization, which should occur only once upon module importation, to optimize subsequent analysis.
    \item The sentiment analyzer must support two modes of text processing: whole text and sentence-segmented; allowing analysis at both document and sentence levels.
    \item The sentiment analyzer must allow the final sentiment attributes to be obtained from the mean or the median of the individual scores.
\end{functionalmod}

\subsubsection{Non-functional requirements}

\newcounter{DSNF}
\begin{nonfunctionalmod}[DS]
    \item The dataset acquirement should be reasonable in both time and size, taking into account \acs{api} rate limits, computer constraints and time dedication.
    \setcounter{DSNF}{\value{enumi}}
\end{nonfunctionalmod}
\subparagraph{Sentiment analyzer}
\begin{nonfunctionalmod}[DS]
    \setcounter{enumi}{\value{DSNF}}
    \item The sentiment analyzer should use optimized, state-of-the-art libraries for text processing, so its usage is viable for real world applications.
\end{nonfunctionalmod}