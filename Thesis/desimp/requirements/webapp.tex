\subsubsection{Functional requirements}
\newcounter{WAF}
\begin{functionalmod}[WA]
    \item The web application will provide a Last.fm track previewer that allows users to listen to music samples from YouTube and Spotify.
    \item The web application will include a text sentiment analyzer that receives text inputs and presents analysis results in a table format, including the extracted sentiment attributes.
    \item A Last.fm user scraper will be part of the showcase tools, allowing users to choose the type and quantity of data to be scraped (tracks, artists, albums, tags), specify the username, and select the data source (\acs{api} or scraped database). The scraped results will be displayed using tables or piecharts to allow easy analysis.
    \item The web application will be able to generate music recommendations using the best models trained on the scraped dataset. Users should be able to select the desired model, set the username, results cutoff, and define the number of results per page.
    \begin{functionalmod}[WA]
        \item If no username is selected and the model is personalized, a random user will be sampled from the database.
        \item Each recommended track will include details such as artist, album, tags, rank and \acs{vad} values, including a link to its preview page using the Last.fm track previewer.
        \item The recommendation results will include a link to the Last.fm user scraper endpoint for the chosen (or random) user, allowing to compare their data with the recommended tracks.
    \end{functionalmod}
    \setcounter{WAF}{\value{enumi}}
\end{functionalmod}
\subparagraph{Recommendation library and models}
\begin{functionalmod}[WA]
    \setcounter{enumi}{\value{WAF}}
    \item The recommendation models must be deployable on the web application, ensuring adequate prediction times when users request recommendations.
    \item The recommendation models must be serializable, as no training should be done for deployed models.
    \item The recommendation models must be capable of providing personalized recommendations for any user in the dataset, based on the chosen user's data.
    \item The context-aware recommendation models must allow incorporation of additional features into the training process, automatically or through configuration settings.
    \item The recommendation library must provide a way to implement custom models, through abstract recommender modules, for instance.
\end{functionalmod}

\subsubsection{Non-functional requirements}
\newcounter{WANF}
\begin{nonfunctionalmod}[WA]
    \item The web application should provide a responsive and intuitive user interface for seamless interaction with the platform's features.
    \item The system should be scalable to handle a growing number of users and accommodate future updates and enhancements.
    \item The application should have low latency, ensuring quick response times to user requests; otherwise, an indication should be shown.
    \item The recommendations generated by the web application should have a user-friendly interface to browse the recommended tracks.
    \setcounter{WANF}{\value{enumi}}
\end{nonfunctionalmod}
\subparagraph{Recommendation library and models}
\begin{nonfunctionalmod}[WA]
    \setcounter{enumi}{\value{WANF}}
    \item The recommendation models should be implemented as independent code modules, to enable easy maintainability and expansion, allowing for any additional functionality or customization of training processes.
    \item The recommendation library should offer flexible and configurable options to select data features and settings for training and evaluation.
    \item The recommendation models should be optimized for efficient training and prediction, using high-performance computing techniques such as \acs{cpu} vectorization or \acs{gpu} parallel computing.
\end{nonfunctionalmod}