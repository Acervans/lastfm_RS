For the development of the web application, the package of choice was the Python-based web framework Django, owing to its extensive feature set and robust \acs{mvt} architecture, which facilitated the development process. Python, being the main programming language used throughout the entirety of the project, was an ideal choice for the backend functionality due to its readability and extensive support for machine learning and data analysis libraries, such as PyTorch~\cite{PYTORCH}, TensorFlow~\cite{TENSORFLOW} and Pandas~\cite{PANDAS}.

Additional technologies were used for the frontend as well, such as Bootstrap 5, jQuery and \acs{ajax}, which make the application feel responsive and dynamic. For instance, the user scraper makes asynchronous, independent \acs{ajax} calls to the server for each of the scraped items, such that the user may freely scroll and view the data obtained at any moment, without having to wait for the entire process.
\vspace{1em}
\begin{figure}[Example music recommendations]{FIG:EXAMPLERECS}{Example music recommendations (see \hyperref[AP:WEBAPP]{Appendix A} for additional screenshots).}\includegraphics[width=1\textwidth]{img/SSRecs.png}
\end{figure}

Figure~\ref{FIG:EXAMPLERECS} presents example recommendations generated for a random user, arranged as a catalogue of tracks. This was possible thanks to Bootstrap's grid layouts, styles and modals.