\subsubsection{Data preprocessing and feature selection}

The data utilized for the experiments consisted of all the users, items and interactions between them, along with various item features, imperatively including \acs{vad} and sentiment ratio. Prior to incorporating any features, preprocessing steps were carried out, such as assigning ratings to interactions, grouping tags by item, reducing sparsity for tracks' tags and sentiment attributes by using those from artists or albums, and normalization (see \hyperref[AP:DATAPROC]{Appendix C} for a detailed depiction). Several libraries were leveraged during this preparation process, including efficient data processing modules like NumPy~\cite{NUMPY} and Pandas, as well as visualization toolkits like Matplotlib~\cite{MATPLOTLIB} and seaborn~\cite{SEABORN}; all integrated into Jupyter Notebooks~\cite{JUPYTER} for interactive data analysis and handling. 

For conducting the experiments, different combinations of features were embedded into RecBole models with configuration files, including tokens (artists, albums and tags), and numerical values (sentiment attributes). Moreover, users' top artists and top albums were not included as user features, since they were tested and did not enhance recommendations, presumably being inferred from tracks' artists and albums. Features unrelated to \acs{vad} were necessary to provide a fair and rigorous performance comparison, as the results from using only sentiment would not be as insightful.

\subsubsection{Feature integration}

The process of feature integration was largely automated, since the context-aware recommender models processed them with PyTorch on initialization, as embeddings for neural networks. In the case of \emph{CosineSimilarityRecommender} model, the tags were accessed in the constructor by Scikit-learn~\cite{SKLEARN}, to build the vectorizer and create the matrix of vectorized tags, and likewise for \emph{HybridVADRecommender} with sentiment features, to store the necessary variables.